%подсказки по LaTeX http://mydebianblog.blogspot.ru/2008/11/latex.html] 
\documentclass[a4paper,14pt]{extreport}

%\documentclass[a4paper,14pt]{report} %размер бумаги устанавливаем А4, шрифт 12пунктов
\renewcommand{\baselinestretch}{1.5} %1,5 междустрочный интервал
\usepackage{ucs} %Зачем этот пакет, если уже utf подключен 

%%Перенес по причине не работы русского текста в математическом режиме
\usepackage{xcolor}	%цветовые настройки гиперссылок
\usepackage{hyperref} %генерация гиперссылок
% Цвета для гиперссылок
\definecolor{linkcolor}{HTML}{799B03} % цвет ссылок
\definecolor{urlcolor}{HTML}{799B03} % цвет гиперссылок
\hypersetup{pdfstartview=FitH,  linkcolor=linkcolor,urlcolor=urlcolor, colorlinks=true}
%Русские символы в математическом режиме
%\usepackage[warn]{mathtext}   
\usepackage[warn]{mathtext}   
\usepackage[T2A]{fontenc}
\usepackage[utf8x]{inputenc} % Включаем поддержку UTF8
%\usepackage{times} % Шрифт Times New Roman
\usepackage[small]{titlesec} % чуть уменьшаем размер заголовков глав и разделов
\usepackage[english, russian]{babel}%используем русский и английский языки с переносами

%%Работа с математическими пакетами
\usepackage{amsmath}

%Для вида (2.1.1)
\numberwithin{equation}{section}
%%Понятия зачем это
\newcommand\normalsubformula[1]{\text{\mathversion{normal}$#1$}}

\usepackage{amsfonts}
\usepackage{amssymb}
\usepackage{cite}
\usepackage{enumerate}
\usepackage{float}

\usepackage{indentfirst} %Включаем первый отступ для абзаца
\usepackage{geometry} % Меняем поля страницы
	\geometry{left=2cm}% левое поле
	\geometry{right=1.5cm}% правое поле
	\geometry{top=1cm}% верхнее поле
	\geometry{bottom=2cm}% нижнее поле

\usepackage[final]{graphicx}
\graphicspath{{images/}}%путь к рисункам
\RequirePackage{caption}
	\DeclareCaptionLabelSeparator{defffis}{ -- }
	\captionsetup{justification=centering,labelsep=defffis}
	\renewcommand{\theenumi}{\arabic{enumi}}% Меняем везде перечисления на цифра.цифра
	\renewcommand{\labelenumi}{\arabic{enumi}}% Меняем везде перечисления на цифра.цифра
	\renewcommand{\theenumii}{.\arabic{enumii}}% Меняем везде перечисления на цифра.цифра
	\renewcommand{\labelenumii}{\arabic{enumi}.\arabic{enumii}.}% Меняем везде перечисления на цифра.цифра
	\renewcommand{\theenumiii}{.\arabic{enumiii}}% Меняем везде перечисления на цифра.цифра
	\renewcommand{\labelenumiii}{\arabic{enumi}.\arabic{enumii}.\arabic{enumiii}.}% Меняем везде перечисления на цифра.цифра

