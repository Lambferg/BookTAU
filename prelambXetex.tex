%Заместо PDFLATEX, используется XeTex, просто для тестирования.
\documentclass[a4paper,14pt]{extreport}
%Xetex
\usepackage{polyglossia}
\setdefaultlanguage[spelling=modern]{russian}
\setotherlanguage{english}

\setmainfont{Times New Roman}%CMU Serif
\setsansfont{Times New Roman}%CMU Sans Serif
\setmonofont{Times New Roman}%CMU Typewriter Text

\usepackage[russian]{hyperref}

%
%\setmathfont{font/latinmodern-math.otf}
%Магия для поялвения кирилицы в формулах
\DeclareSymbolFont{cyrletters}{\encodingdefault}{\familydefault}{m}{it}
\newcommand{\makecyrmathletter}[1]{%
	\begingroup\lccode`a=#1\lowercase{\endgroup
		\Umathcode`a}="0 \csname symcyrletters\endcsname\space #1
}
\count255="409
\loop\ifnum\count255<"44F
\advance\count255 by 1
\makecyrmathletter{\count255}
\repeat
%Xetex

%\documentclass[a4paper,14pt]{report} %размер бумаги устанавливаем А4, шрифт 12пунктов
\renewcommand{\baselinestretch}{1.5} %1,5 междустрочный интервал
\usepackage{ucs}
%\usepackage[T2A]{fontenc}
%\usepackage[utf8x]{inputenc} % Включаем поддержку UTF8
%\usepackage{times} % Шрифт Times New Roman
\usepackage[small]{titlesec} % чуть уменьшаем размер заголовков глав и разделов
%\usepackage[english,russian]{babel}%используем русский и английский языки с переносами
\usepackage{amssymb,amsfonts,amsmath,mathtext,cite,enumerate,float}
\usepackage{indentfirst} %Включаем первый отступ для абзаца
\usepackage{geometry} % Меняем поля страницы
\geometry{left=2cm}% левое поле
\geometry{right=1.5cm}% правое поле
\geometry{top=1cm}% верхнее поле
\geometry{bottom=2cm}% нижнее поле
\usepackage[final]{graphicx}
\graphicspath{{images/}}%путь к рисункам
\RequirePackage{caption}
\DeclareCaptionLabelSeparator{defffis}{ -- }
\captionsetup{justification=centering,labelsep=defffis}
\renewcommand{\theenumi}{\arabic{enumi}}% Меняем везде перечисления на цифра.цифра
\renewcommand{\labelenumi}{\arabic{enumi}}% Меняем везде перечисления на цифра.цифра
\renewcommand{\theenumii}{.\arabic{enumii}}% Меняем везде перечисления на цифра.цифра
\renewcommand{\labelenumii}{\arabic{enumi}.\arabic{enumii}.}% Меняем везде перечисления на цифра.цифра
\renewcommand{\theenumiii}{.\arabic{enumiii}}% Меняем везде перечисления на цифра.цифра
\renewcommand{\labelenumiii}{\arabic{enumi}.\arabic{enumii}.\arabic{enumiii}.}% Меняем везде перечисления на цифра.цифра
\usepackage{xcolor}	%цветовые настройки гиперссылок
\usepackage{hyperref} %генерация гиперссылок
% Цвета для гиперссылок
\definecolor{linkcolor}{HTML}{799B03} % цвет ссылок
\definecolor{urlcolor}{HTML}{799B03} % цвет гиперссылок

\hypersetup{pdfstartview=FitH,  linkcolor=linkcolor,urlcolor=urlcolor, colorlinks=true}
%%Для кирилицы в формулах
\usepackage{unicode-math} 
\setmathfont{font/xits-math.otf}
