\chapter{Тест}
\section{Введение в теорию автоматического управления}
На рубеже XVIII-XIX веков в эпоху промышленного переворота в Европе начинается новый этап развития автоматики, связанный с внедрением ее в промышленность. 1765 год знаменуется постройкой регулятора уровня котла паровой машины И.И. Ползунова. В 1784 го-ду появляется центробежный регулятор скорости паровой машины Дж.Уатта.

В это время формируется ряд важных принципов автоматики: принцип регулирования по отклонению Ползунова - Уатта и принцип регулирования по нагрузке Понселе. Первый из них развился в кон-цепцию обратной связи, второй - в теорию инвариантности (Г.В. Щи-панов, Н.Н. Лузин, Б.Н. Петров). Идея регулирования по нагрузке мо-жет быть проиллюстрирована на примере генератора с последова-тельным (сериесным) возбуждением (рис.1.1). При изменении нагруз-ки меняется ток возбуждения, который соответствующим изменением магнитного потока компенсирует дополнительное падение напряже-ния на внутреннем сопротивлении якоря генератора. Однако если при этом по каким-либо причинам изменяется скорость вращения якоря генератора, то застабилизировать напряжение на нагрузке в этой схеме уже не удается.
\newpage
От этого недостатка свободна схема, приведенная на рис. 1.2, - именно вследствие использования принципа обратной связи. В этой схеме входной потенциометр служит для задания (коэффициент  ) величины стабилизируемого напряжения; потенциометр, подключен-ный к якорю генератора, позволяет регулировать коэффициент об-ратной связи  . В этом случае, в отличие от систем регулирования по возмущению, не важно, какая именно причина вызвала изменение регулируемой величины. При изменении напряжения на щётках гене-ратора в соответствии с электрической  схемой изменяется напряже-ние на обмотке возбуждения. При отрицательном знаке обратной свя-зи знак приращения напряжения возбуждения противоположен знаку изменения напряжения якоря генератора. В итоге результирующая величина отклонения напряжения генератора уменьшается по срав-нению с соответствующим уходом напряжения в системе без обрат-ной связи. 

%PIC вставить


%PIC вставить
\begin{equation}
\Delta l_{зол}=K_{1}\cdot\Delta l_{1}-K_{2}\cdot\Delta l_{2}
\end{equation}

$\Delta F_{пр}=K_{пр}\cdot\Delta l_{2}  $

%%Проверка работоспособности 
\begin{equation}
	Тест \quad русского \quad языка
\end{equation}


\begin{equation}
\left(1+K_{зол}K_{2}\cfrac{T_{д}}{T_{д}p+1}\right)\cdot p\Delta l_{сц}=K_{зол}K_{1}\Delta l_{1}
\end{equation}
